
\section{Making an initial density field}

The density field of a cosmological simulation can be made with the use of the previously made pseudo random number generator. A Gaussian random field is made in Fourier space. The dispersion of the gaussian field is given by,

\begin{equation*}
\sigma^2 = P(k) \sim k^n
\end{equation*}
For this exercise it is assumed that the normalization constant is equal to 1 such that $\sigma^2 = k^n$. The fourier space matrix that is constructed consists of complex entries that are drawn from the gaussian distribution with mean zero, $\mu = 0$, and variance equal to the dispersion, $\sigma^2 = k^n$. Here $k$ are the wavenumbers and $n$ determines how steep the power spectrum will be in loglog space. The matrix that is constructed has a special symmetry, namely $\bar{Y}(-\textbf{k}) = \bar{Y}^*(\textbf{k})$ (i.e. the elements in the fourier- matrix are symmetric). This symmetry has to hold such that the inverse fourier transform generates real numbers, instead of complex numbers. The symmetry is obtained in the following way. If $M_{k_x,k_y}$ are the randomly created complex values in the matrix, with $k_x$ and $k_y$ being the wavevectors containing the wavenumbers, then the following symmetries should hold: $M_{k_x,k_y} = M_{-k_x,-k_y}^*$. However, if $k_x = 0$ and $k_y = 0$ then we assign the value $0 + 0i$ to this matrix element. This is because of the power $n$. If this power is negative then this matrix element will go to infinity for $k_x,k_y = 0$. Then we also have to take care of the so-called nyquist frequency $k_{\mathrm{nyq}}$. The nyquist frequency element does not have a corresponding negative wavenumber. Therefore the only way to have a complex value that equals its non-negative conjugate, it to assign only a real-value to this matrix element. Therefore, $k_x = k_y = k_{\mathrm{nyq}} = \mathbb{R}$. Another property of the nyquist frequency is $k_{\mathrm{nyq}} = -k_{\mathrm{nyq}}$. This ensures that if either $k_x = k_{\mathrm{nyq}}$ or $k_y = k_{\mathrm{nyq}}$, the matrix elements do have a complex conjugate value. The only exception in this particular row and column is whenever $k_x = 0$ and $k_y = k_{\mathrm{nyq}}$ (or the other way around), then we have the same problem as when having two nyquist frequencies again. Therefore, these matrix elements have to be real-valued as well. When the symmetric matrix has been created, the inverse fourier transform is calculated with the use of a scipy package. This package normalizes the inverse fourier transform by dividing it by the number of gridpoints $N$ (in 2d). For this exercise a minimal physical size has also been set. The wave-vector $\textbf{k}$ is affected by this minimal physical size, because all its elements are divided by the maximal physical size. If the minimal physical size is $s_{\mathrm{min}}$, then the maximal physical size is $s_{\mathrm{max}} = s_{\mathrm{min}}\cdot N$, where $N$ is the number of gridpoints. So in the cases above the maximum physical size is 1024 Mpc, since there are 1024 gridpoints and the minimal physical size is set to 1 Mpc. The wavenumbers are given by $k = \frac{2\pi}{\lambda}$, where $\lambda$ is a physical size of an object. So for a wave-vector with entries $k = 0, \pm 1, \pm 2, ... \frac{N}{2}$, we get as minimum wavenumber $k_{\mathrm{min}} = 1\cdot\frac{2\pi}{s_{\mathrm{max}}}$, with $s_{\mathrm{max}}$ being the maximum physical size. Then for the maximum wavenumber we get $k_{\mathrm{max}} = \frac{N}{2}\cdot \frac{2\pi}{s_{\mathrm{max}}} = \frac{N\pi}{s_{\mathrm{max}}} = \frac{N\pi}{s_{\mathrm{min}}N} = \frac{\pi}{s_{\mathrm{min}}}$.
The code that is used to produce such a matrix that has the correct fourier-symmetry and wavenumbers is given below\footnote{Note that the RNG is also used in this exercise but is not shown here since it has already been shown.},

\lstinputlisting{Q2.py}

The output of the code are three plots for different values of $n$ ($n = -1$, $n = -2$, and $n = -3$). Each plot is created with the exact same random numbers generated by the PRNG described in the previous section. The plots are the inverse fourier transform of the symmetric gaussian filled matrix, which create an initial density field that is gaussian randomly distributed. 

\begin{figure}[h]
\vspace{-1.2em}
\centering
\includegraphics[scale=0.6]{plots/density_field_1.png}
\vspace{-1.5em}
\caption{Gaussian random field for $n = -3$.}
\label{n_3}
\end{figure} 
\begin{figure}[h]
\vspace{-2.5em}
\centering
\includegraphics[scale=0.6]{plots/density_field_2.png}
\vspace{-1em}
\caption{Gaussian random field for $n = -2$.}
\end{figure} 
\begin{figure}[h]
\centering
\includegraphics[scale=0.6]{plots/density_field_3.png}
\caption{Gaussian random field for $n = -1$.}
\end{figure} 
Each of the figures are quite different. For all three power spectra ($n=-1$, $n=-2$, $n=-3$), the dispersion of the gaussian field differs. For $n=-3$ this dispersion is biggest since $k$ is always equal to or smaller than 1. This means that we should see a lot of structure for this particular random gaussian field. Taking a look at the figures, this indeed seems to be the case. When the value for $n$ increases, the amount of structure seen in the density field seem to disappear and it becomes more and more a homogeneous field. 
